\chapter{Qualitätszielbestimmung}

\begin{itemize}
	\item Robustheit: wichtig
	\item Zuverlässigkeit: sehr wichtig
	\item Korrektheit: sehr wichtig
	\item Benutzerfreundlichkeit: sehr wichtig
	\item Effizienz: wichtig
	\item Echtzeitfähigkeit: unwichtig
	\item Portierbarkeit: unwichtig
	\item Kompatibilität: wichtig
\end{itemize}

Der Grund für die Existenz von ewok ist die mangelnde Benutzerfreundlichkeit der vorigen Lösungen.
Daher hat Benutzerfreundlichkeit für ewok hohe Prioriät.
Doch dies allein kann für einen Package-Manager nicht genügen, da sowohl die zuverlässige, als auch die korrekte Installation von Paketen Benutzerfreundlichkeit erst ermöglichen. Darüber hinaus kann nur durch ausreichende Robustheit die Verwendung im Alltag garantiert werden. Diese Robustheit muss angesichts der großen zu transferierenden Datenmengen durch Effizienz komplementiert werden.
Kompatibilität ist in soweit wichtig, als dass ewok unter einer Vielzahl unterschiedlicher GNU/Linux-Distributionen laufen soll. Da GNU/Linux aber das einzige Zielsystem ist, ist Portierbarkeit unwichtig, obgleich dennoch durch Python gewährleistet.
